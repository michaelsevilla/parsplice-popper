\documentclass[sigconf]{acmart}
\usepackage{arydshln}

\usepackage{booktabs} % For formal tables


% Copyright
\setcopyright{none}
\settopmatter{printacmref=false}
%\setcopyright{acmcopyright}
%\setcopyright{acmlicensed}
%%\setcopyright{rightsretained}
%\setcopyright{usgov}
%\setcopyright{usgovmixed}
%\setcopyright{cagov}
%\setcopyright{cagovmixed}


%% DOI
%\acmDOI{10.475/123_4}
%
%% ISBN
%\acmISBN{123-4567-24-567/08/06}
%
%%:Conference
\acmConference[]{}{}{}
%\copyrightyear{2016}
%
%\acmArticle{4}
%\acmPrice{15.00}

% These commands are optional
%\acmBooktitle{Transactions of the ACM Woodstock conference}
%\editor{Jennifer B. Sartor}
%\editor{Theo D'Hondt}
%\editor{Wolfgang De Meuter}


\begin{document}
\title{Towards Dynamic Load Balancing Policies\\in Software-Defined Storage}
%\titlenote{Produces the permission block, and
%  copyright information}
%\subtitle{Extended Abstract}
%\subtitlenote{The full version of the author's guide is available as
%  \texttt{acmart.pdf} document}

\author{Michael A. Sevilla, Carlos Maltzahn}
\affiliation{\institution{University of California, Santa Cruz}}
\email{{msevilla, carlosm}@soe.ucsc.edu}

\author{Brad Settlemyer, Danny Perez, David Rich, Galen Shipman (LANL)}
\email{{bws, danny_perez, dor, gshipman}@lanl.gov}

\begin{abstract}

Using a detailed keyspace analysis we show the need for a distributed, load
balancing key-value store for the ParSplice molecular dynamics simulation.  To
enable this, we use the Mantle framework to change load balancing policies and
to quantify performance. We then show the inefficiencies of load balancing
using a single policy by designing a two-part, dynamic policy that achieves
93\% efficiency with less than 1\% of the memory usage. Finding this policy was
tedious and done by hand, so we also present a machine learning approach that
has the potential to feed into our load balancing policy engine.  The paper
shows the usefulness of the Mantle approach for key-value store workloads and
it our hope that the initial steps we have taken designing dynamic load
balancing policies jump starts the community along similar efforts.

\end{abstract}

%
% The code below should be generated by the tool at
% http://dl.acm.org/ccs.cfm
% Please copy and paste the code instead of the example below. 
%
%\begin{CCSXML}
%<ccs2012>
% <concept>
%  <concept_id>10010520.10010553.10010562</concept_id>
%  <concept_desc>Computer systems organization~Embedded systems</concept_desc>
%  <concept_significance>500</concept_significance>
% </concept>
% <concept>
%  <concept_id>10010520.10010575.10010755</concept_id>
%  <concept_desc>Computer systems organization~Redundancy</concept_desc>
%  <concept_significance>300</concept_significance>
% </concept>
% <concept>
%  <concept_id>10010520.10010553.10010554</concept_id>
%  <concept_desc>Computer systems organization~Robotics</concept_desc>
%  <concept_significance>100</concept_significance>
% </concept>
% <concept>
%  <concept_id>10003033.10003083.10003095</concept_id>
%  <concept_desc>Networks~Network reliability</concept_desc>
%  <concept_significance>100</concept_significance>
% </concept>
%</ccs2012>  
%\end{CCSXML}

%\ccsdesc[500]{Computer systems organization~Embedded systems}
%\ccsdesc[300]{Computer systems organization~Redundancy}
%\ccsdesc{Computer systems organization~Robotics}
%\ccsdesc[100]{Networks~Network reliability}
%
%
%\keywords{ACM proceedings, \LaTeX, text tagging}


\maketitle

\section{Introduction}

Key-Value stores scale because they support (1) fine scale annotation and (2)
flexible, extensible formats. Science applications are structured, entropy
increases over time (e.g., Figure~\ref{fig:futurework-regimes} shows key
distribution and key popularity changing over time). Our hypothesis is that
re-distributing keys requires dynamic load balancing policies, similiar to
distributed file systems.  A one-size-fits all policy is not sufficient. Our
contributions are:

\begin{enumerate}
  \item an analysis of the ParSplice keyspace
  \item using a modern distributed key-value store
  \item positive effects of Mantle 
\end{enumerate}

\begin{figure}[t]
  \noindent\includegraphics[width=0.5\textwidth]{figures/futurework-regimes.png}\\
  \caption{Caching the 100 most recently accessed keys is sufficient for run 1
  but more active keyspaces like run 0 need dynamic load balancing
  policies.\label{fig:futurework-regimes}}
\end{figure}


%% What is the problem?
%Load balancing is a useful tool for optimizing performance in systems that
%service highly accessed data\footnote{In this paper, we use the term ``data" to
%refer to the partitioned key-value pairs AND file system metadata.} but
%deciding how to make the migrations is a risky trade-off. In this paper, we
%show that a one-size-fits-all data load balancing policy is not sufficient for
%even the simplest of HPC applications and argue for a dynamic load balancing
%policy.
%
%% Explain techniques
%Resource migration is the key mechanism for load balancing. In storage, data
%can be distributed to alleviate overloaded servers or it can be concentrated to
%exploit locality. These techniques are at odds and selecting the wrong
%technique can have catastrophic consequences. For example, migrating data to an
%already overloaded server or increasing the network hops by spreading data
%across an underutilized cluster will impact performance negatively.
%
%% Why is concentration vs. distribution difficult?
%Unfortunately, deciding which optimization to use is difficult to reason about,
%especially with the scale and complexity of today's HPC architectures. While
%the mechanisms are usually built into the systems, the policies often times
%less refined and much more sensitive to the workload. So a system may have the
%ability exploit locality using techniques like bulk operations, multiple
%partition strategies, secondary indexes, and caching but deciding when, where,
%and how to use them is workload dependent and difficult to figure out.
%
%% What we did in the paper
%This paper takes an API designed to migrate file system metadata and applies it
%to an HPC key-value store.  The API helps control distribution and
%concentration by letting the administrator define how to migrate load, where to
%migrate load, and how much load to migrate. While designed for a different
%domains, this API encompasses many of the same properties we need for an HPC
%key-value store, namely:
%
%\begin{itemize}
%  \item services small/frequent requests
%  \item popularity drives distribution
%  \item locality drives concentration
%\end{itemize}
%
%\begin{figure}[t]
%  \noindent\includegraphics[width=19pc,angle=0]{figures/arch-parsplice.png}\\
%  \caption{ParSplice is a ready-heavy HPC application where producers use a
%  database for consistency. Replacing the single-node database with HXHIM
%  improves performance with load balancing.
%  \label{fig:arch-parsplice}}
%\end{figure}
%
%% Why is HXHIM a good fit?
%To show the efficacy of this approach, we examine the ParSplice molecular
%dynamics simulation application shown in Figure~\ref{fig:arch-parsplice}.
%ParSplice uses a single-node database for consistency, where producers, \(P\),
%push and pull coordinates, \{\(x_i, y_i\)\}, based on the segments,
%\{\(s_i\)\}, assigned by the splicer, \(S\). In this paper, we replace the
%database with a distributed key-value store designed for HPC enjoy performance
%optimizations for:
%
%\begin{itemize}
%  \item \texttt{put()} because of the distributed sync and load balancing based on:
%  \begin{itemize}
%    \item lazy synchronization with tombstones and RPCs
%    \item strong synchronization with consensus and blocking
%  \end{itemize}
%  \item \texttt{get()} because of the load balancing
%\end{itemize}
%
%It has 4 phases:
%
%\begin{enumerate}
%
%  \item splicer (S) tells producers (P) to compute segments for state \(s_i\)
%
%  \item P's pull initial coordinates \{\(x_i, y_i\)\} from database
%
%  \item a P inserts completed coordinates for segment \(s_i\) into database and
%  S broadcasts next segment(s) \(s_j\) 
%
%  \item P's pull new segment coordinates \{\(x_j, y_j\)\}
%\end{enumerate}
%
%, which has both a high computational footprint and data locality.
%The former suggests distribution to avoid hot spots while the latter encourages
%concentration to leverage the database's secondary indeices, bulk operations,
%and key redistribution functionality. 
%
%% Why is HXHIM a good fit?
%To show this approach at scale we study
%ParSplice~\cite{perez:jctc20150parsplice}, an HPC dynamics simulator that has
%both a high computation footprint, which suggests distribution to avoid hot
%spots, and data locality, which alternatively encourages concentration so the
%key-value store can use its functionality for secondary indices, bulk
%operations, and key redistribution. ParSplice uses both molecular dynamic (MD)
%and accelerated molecular dynamic methods (AMD) for simulations with long
%periods of inactivity and short periods of ``interesting" events.  Molecules in
%periods with many events are simulated with MD methods, which are exact but can
%only be run for a fixed, short period of time because the cumulative error
%grows so large. Alternatively, longer trajectories are simulated with AMD
%methods, which use statistics and parallelization to show the less precise
%state-to-tate trajectories. ParSplice tackles ``low-barrier problems", where
%the types of energy barriers separating states of the system are non-uniform
%(i.e. some require less energy than others). It chops long trajectories into
%parallelizable units called segments, where the segments can also be spliced
%together to form longer trajectories; this approach allows ParSplice to trade
%off accuracy for speed in a configurable way.
%
%% How is ParSplice implemented?
%ParSplice stores segments in a database while it runs. The splicer pastes
%segments generated by \(n\) producers.
%
%% What do we contribute?
%In this paper, we make the following contributions:
%
%\begin{enumerate}
%
%  \item protype that controls concentration and distribution using the bulk
%  operations, secondary indicies, and cursor types mechanisms
%  from~\cite{greenberg:hotstorage2015-mdhim}. 
%
%  \item quantifies benefits of server/client-side caching, many small messages,
%  and bulk operations.
%
%\end{enumerate}

\section{Background}

\subsection{ParSplice}
\label{sec:parsplice}

\begin{figure}[t]
  \noindent\includegraphics[width=19pc,angle=0]{figures/arch-parsplice.png}\\
  \caption{ParSplice is a ready-heavy HPC application where producers use a
  database for consistency. Replacing the single-node database with HXHIM
  improves performance with load balancing.
  \label{fig:arch-parsplice}}
\end{figure}


ParSplice~\cite{perez:jctc20150parsplice} is a molecular dynamics simulation
developed at LANL. It has 4 phases, as depicted in Figure~\ref{fig:arch-parsplice}:

\begin{enumerate}

  \item splicer (S) tells producers (P) to compute segments for state \(s_i\)

  \item P`s pull initial coordinates \{\(x_i, y_i\)\} from database

  \item a P inserts completed coordinates for segment \(s_i\) into database and
  S broadcasts next segment(s) \(s_j\) 

  \item P`s pull new segment coordinates \{\(x_j, y_j\)\}

\end{enumerate}

The database is single node (LevelDB or BerkeleyDB) with caches in front. Our
goal is to replace this architecture with a distributed key-value store to
solve the immediate sync problems that the ParSplice team is facting. Sliding
in something like MDHIM~\cite{greenberg:hotstorage2015-mdhim} has the added
benefit of enabling load balancing. ParSplice has distinct workload phases and
a well-known keyspace (Figure~\ref{fig:methodology-keyspace}) so its
performance should improve with better load balancing.

\subsection{MDHIM: Key-Value Store for HPC}
\label{sec:hxhim}

\begin{figure}[tb]
  \noindent\includegraphics[width=19pc,angle=0]{figures/arch-hxhim.png}\\
  \caption{The MDHIM architecture.}
  \label{fig:arch-hxhim}
\end{figure}

% What is MDHIM
MDHIM is a key-value store designed for HPC architectures and multi-dimensional
data. It is based off MDHIM~\cite{greenberg:hotstorage2015-mdhim}, the
multi-dimensional indexing middleware. Figure~\ref{fig:arch-hxhim} has a crude
sketch of the MDHIM architecture. Each MPI rank has an instance of the
application, which has the client library linked in. An MPI rank can also have
a ``range server", which stores the key-value pairs in a local databse (either
LevelDB or MySQL). Data is located with a consistent hash, which is
configurable.

% What are the indexes?
The primary and secondary indices shown on the right side of
Figure~\ref{fig:arch-hxhim} are views of the data that the range server
manages.  The primary index is the same hash used by the global partitioner.
The secondary index or indices are user-defined tables organized in a different
way from the primary index. The goal of the secondary indices is to speed up
queries that need to aggregate dat ({\it e.g.} find the maximum values). In the
example, the range server and the key in the primary index is located with a
hash of the mesh location. The secondary index is organized by pressure, so
queries asking for a certain atmosphere can be serviced in O\(1\), consisting
of one lookup in the pressure index and one lookup into the primary index.

% Why is it tailored to HPC?
MDHIM tailors its mechanisms and policies to HPC, showing improved performance
over cloud-based key-value stores like Cassandra. It has cursor types for
walking the key-value store, bulk operations for exploiting data locality,
per-job server spawning, and pluggable backends for its local database and
network type (infiniband/RDMA). Its policies are flexible, supporting
customized partitioning strategies and user-defined secondary indices. This
allows the system to choose whether to send load to the client or server.

\subsection{Comparing Mantle and MDHIM}
\begin{table*}
\centering
\begin{tabular}[tb]{ r | l | l | l | l }
                       & 
                       & \multicolumn{1}{c|}{\centering Both} 
                       & \multicolumn{1}{c|}{\centering Mantle/CephFS}
                       & \multicolumn{1}{c}{\centering MDHIM}
                       \\\hline
  workload   & characteristics     & small/frequent requests  & data access            & data management \\
             & write-intensive     & partition across cluster & fragment directories*  & NOT IMPLEMENTED \\
             & read-intensive      & replicate across cluster & copy directories*      & NOT IMPLEMENTED \\\hdashline
  system     & measure workload    & yes                      & directory temperature  & range server counts \\
  mechanisms & measure utilization & yes                      & CPU, network, memory   & range server buffer size \\
             & migrate resources   & almost                   & \texttt{export\_dir()} & \texttt{mdhimB}\{\texttt{Get}, \texttt{Put}\}\texttt{()} \\
             & partition resources & yes                      & subtrees \& dirfrags   & secondary index, cursor type, bulk operations\\\hdashline
  migration  & interval            & configurable             & every 10 seconds       & every query \\
  decisions  & global state        & decentralized decisions  & heartbeats for metrics & NOT IMPLEMENTED \\
  \multicolumn{5}{c}{}\\
  \multicolumn{5}{c}{\tiny *Mechanisms implemented in CephFS, not integrated into Mantle}
\end{tabular}
\caption{Comparing the design goals and implementatons of Mantle and MDHIM.}
\label{fig:arch-comparison}
\end{table*}

% Why are the designed for the same type of workload?
The ``Both" column of Table~\ref{fig:arch-comparison} shows how Mantle and
MDHIM have similar designs. The workloads are very similar as the the services
respond to small and frequent requests, which results in hot spots and flash
crowds. As a result, popularity of the data, not the size, drives distribution
in both systems. Both workloads also have data locality so the systems have
mechanisms for leveraging requests with similar semantic meaning.  Finally, the
overall design of both systems is decentralized meaning that there is no
centralized scheduler and each server has an inconsistent global view.

% What are the challenges?
Despite the similarities, integrating the Mantle API with MDHIM has both design
and technical challenges. Mantle is reactive to the workload as opposed to
MDHIM migrations, which are triggered based on the request type. As a result,
Mantle has functionality for exchanging server utilization (CPU, network,
memory) and workload (tracks request types). MDHIM 


%This paper takes the API and load balancers designed in
%Mantle~\cite{sevilla:sc15-mantle}, the programmabile file system metadata load
%balancer for Ceph, and applies them to
%MDHIM~\cite{greenberg:hotstorage2015-mdhim}, the distributed key-value store
%designed for HPC.



Results should show, In order from most likely to least likely:

\begin{enumerate}

  \item HPC key-value store workloads are structured (because they are mostly
  workflows and simulations) that their job phases can be learned and exploited
  using dynamic load balancing policies.

  \item HPC key-value store workloads are so structured that one
  policy-fits-all

  \item HPC key-value store workloads are not structured enough to be learned

  \item HPC key-value store workload hotspots/flash crowds are too fast to be
  exploited

\end{enumerate}

\section{ParSplice Keyspace Analysis}
\label{sec:parsplice-keyspace-analysis}
%IT HAS 8K keys!  How did we take these measurements

We instrumented ParSplice with performance counters and keyspace counters.  The
performance counters track ParSplice progress while keyspace counters track
which keys are being accessed by the ParSplice ranks. Because the keyspace
counters have high overhead we only turn them on for the keyspace analysis.
The cache hierarchy was unmodified but for the back-end persistent database, we
replaced BerkeleyDB on NFS with LevelDB on Lustre. Original ParSplice
experiments showed that BerkeleyDB's syncs caused reads/writes to bottleneck on
the persistent database node. We also use Riak's customized
LevelDB\footnote{https://github.com/basho/leveldb} version, which comes
instrumented with its own set of performance counters.

All experiments ran on Trinitite, a Cray XC40 with 32 Intel Haswell 2.3GHz
cores per node.  Each node has 128GB of RAM and our goal is to limit the size
of the database to 3\% of RAM\footnote{Empirically, this is a threshold that we
find to work well for most applications}. Note that this is an addition to the
30GB that ParSplice uses to manage other ranks on the same node.  A single Cray
node produced trajectories that are \(5\times\) times longer than our 10 node
CloudLab clusters and \(25\times\) longer than UCSC's 10 node cluster. As
a result, it reaches different job phases faster and gives us a more
comprehensive view of the workload. The performance gains compared to the
commodity clusters have more to do with memory/PCI bandwidth than network.

\begin{figure}[t]
  \noindent\includegraphics[height=4.5cm,width=0.4\textwidth]{figures/methodology-keyspace.png}\\
  \caption{The keyspace size is small but must satisfy many reads as workers
  calculate new segments. Based on these trends, it is likely that we will need
  more than one node to manage segment coordinates when we scale the system or jobs up.
  \label{fig:methodology-keyspace}}
\end{figure}

\begin{figure}[t]
  \noindent\includegraphics[height=4.5cm,width=0.4\textwidth]{figures/methodology-keys.png}\\
  \caption{The keyspace imbalance is due to workers generating deep
  trajectories and reading the same coordinates. Over time, the accesses get
  dispersed across different coordinates resulting in some keys being more
  popular than others.\label{fig:methodology-keys}}
\end{figure}

\subsubsection*{Scalability} Figure~\ref{fig:methodology-keyspace} shows the
keyspace size (black annotations) and request load (bars) after a one hour run
with a different number of workers (\(x\) axis). While the keyspace size and
capacity is relatively modest the memory usage scales linearly with the number
of workers. This is a problem if we want to scale to Trinitite's 6000 cores.
Furthermore, the size of the keyspace also increases linearly with the length
of the run.  Extrapolating these results puts an 8 hour run across all 100
Trinitite nodes at 20GB for the cache.  This puts the memory usage far above
the 3\% threshold we set earlier, even without factoring in the memory usage
from other workers.

\subsubsection*{An active but small keyspace} The bars in
Figure~\ref{fig:methodology-keyspace} show \(50-100\times\) as many reads
(\texttt{get()}) as writes (\texttt{put()}).  Worker tasks read the same key
for extended periods because the trajectory segment is stuck in a superbasin
composed of multiple, similar local minima, so many coordinates are needed before the trajectory
moves on. Writes only occur for the final state of segments generated by worker
tasks; their magnitude is smaller than reads because the caches ignore
redundant write requests. The number of read and write requests are highest at
the beginning of the run when worker tasks generate segments for the same
state, which is computationally cheap (this motivates
Section~\S\ref{sec:static-load-balancing}).

\subsubsection*{Entropy increases over time} The reads per second in
Figure~\ref{fig:motivation-regimes} show that the number of requests decreases
and the number of active keys increases over time. The resulting key access
imbalance for the two growth rates in Figure~\ref{fig:motivation-regimes} are
shown in Figure~\ref{fig:methodology-keys}, where reads are plotted for each
unique state (\(x\) axis). Keys are more popular than others (up to
\(5\times\)) because worker tasks start generating states with different
coordinates later in the run (this motivates
Section~\S\ref{sec:the-need-for-dynamic-load-balancing-policies}).

\subsubsection*{Entropy growth is structured} The access patterns reflect the
locality of computation: worker tasks stuck in superbasins generate segments with
similar coordinates. The growth rate, temperature, and number of workers
changes that locality, which has an effect on the structure of the keyspace.
Figure~\ref{fig:methodology-keys} shows that the number of reads changes with
different growth rates, but that spatial locality is similar ({\it e.g.}, some
keys are still \(5\times\) more popular than others).
Figure~\ref{fig:motivation-regimes} shows how entropy for different growth
rates has temporal locality, as the reads per second for \(\Delta_2\) looks like the
reads per second for \(\Delta_1\) stretched out along the time axis.  Trends also exist
for temperature and number of workers but are omitted here for space. This
structure means that we can learn the regimes and adapt the storage system to
it (this motivates Section~\S\ref{sec:ml-for-the-keyspace}).

\section{Static Load Balancing}
\label{sec:static-load-balancing}

% technical details
In the original ParSplice implementation, each cache node uses as an unlimited
amount of memory to store segment coordinates. We limit the size of the cache
using an LRU eviction policy, where the penalty for a cache miss is retrieving
the data from the persistent database.  We evict keys (if necessary) at every
operation instead of when segments complete because the cache fills up too
quickly otherwise.

% results: cache size trade-offs
\begin{figure}[t]
  \noindent\includegraphics[height=4.5cm,width=0.4\textwidth]{figures/methodology-tradeoff.png}\\
  \caption{The performance and resource utilization trade-off for different
  cache sizes, which are enumerated along the \(x\) axis. ``Baseline" is
  ParSplice unmodified and the ``Static  Policies" limit the size of the
  cache to demonstrate the memory savings of smaller keyspaces on cache nodes.
  \label{fig:methodology-tradeoff}}
\end{figure}

The results for different cache sizes for a growth rate of \(Delta_1\) over a 2.5 hour
run is shown in Figure~\ref{fig:methodology-tradeoff}.  ``Baseline" is the
performance of unmodified ParSplice  measured in trajectory duration
(\(y\) axis) and utilization is measured with memory footprint (\(y2\) axis) of
just the cache.  ``Static Load Balancing Policies" shares the \(y\) axis and
shows the trade-off for different cache sizes. The error bars are the standard
deviation of 3 runs. 

% results: raw numbers
Although the keyspace grows to 150K, a 100K key cache achieves 99\% of the
performance. Decreasing the cache degrades performance and predictability.
While this result is not unexpected, it nonetheless achieves our goal of
showing the benefits of load balancing keys across nodes and that smaller
caches on each node are an effective way to save memory without completely
sacrificing performance.

\section{The Need for Dynamic Load Balancing Policies}
\label{sec:the-need-for-dynamic-load-balancing-policies}

% why is the performance lower for smaller caches?
Despite the memory savings, our results suggest that dynamic load balancing
policies could save even more memory.  Figure~\ref{fig:methodology-tradeoff}
show that a 100K key cache is sufficient as a static policy but the top graph
in Figure~\ref{fig:motivation-regimes} indicates that the cache size could be
much smaller. That graph shows that the beginning of the run is characterized
by many reads to a small set of keys and the end sees much lower reads per
second to a larger keyspace. Specifically, it shows only about 100 keys as
active in the latter half of the run, so a smaller cache should indeed suffice. 

After analyzing traces, we see that the 100 key cache is insufficient because
LevelDB cannot service the read-write traffic. By limiting the size of the
cache, some reads must traverse up the ParSplice cache hierarchy to the
persistent database.  According to Figure~\ref{fig:motivation-regimes}, the
read requests arrive at 750 reads per second in addition to the writes that
land in each tier (about 300 puts/second, some redundant). This traffic
triggers a LevelDB compaction and reads block, resulting in very slow progress.
Traces verify this hypothesis and show reads getting backed up as the
read/write ratio increases. To recap, small caches incur too much load on the
persistent database  at the beginning of the run but smaller caches should
suffice after the initial read flash crowd passes because the keyspace is far
less active. This suggests a two-part load balancing policy.

% what is mantle
To explore dynamic load balancing policies ({\it i.e.} policies that change
during the run), we use the Mantle approach.  Mantle is a framework built on the
Ceph file system that lets administrators control file system metadata load
balancing policies. The basic premise is that load balancing policies can be
expressed with a simple API consisting of ``when", ``where", and ``how much".
The succinctness of the API lets users inject multiple, dynamic policies.

% Why is this a good idea
Although ParSplice does not use a distributed file system, its workload is very
similar because the minima key-value store responds to small and frequent
requests, which results in hot spots and flash crowds.  Distributed file
systems try to find optimal ways to measure, migrate, and partition metadata
load and modern research file systems have shown large performance gains and
better scalability~\cite{zheng:pdsw2014-batchfs, zheng:pdsw2015-deltafs,
grider:pdsw2015-marfs, ren:sc2014-indexfs, patil:fast2011-giga+,
brandt:msst2003-lh}.  Previous work quantified the speedups achieved with
Mantle and formalized balancers that were good for file systems.

\begin{figure}[t]
  \noindent\includegraphics[height=4.5cm,width=0.4\textwidth]{figures/methodology-tradeoff-dynamic.png}\\
  \caption{The performance and resource utilization trade-off of using a
  dynamic load balancing policy that switches to a smaller cache after absorbing
  the initial burstiness of the workload. The sizes of these smaller caches are
  on the \(x\) axis.  \label{fig:methodology-tradeoff-dynamic}}
\end{figure}

Figure~\ref{fig:methodology-tradeoff-dynamic} shows the results of using the
Mantle API to program a dynamic load balancing policy into
ParSplice:

\begin{itemize}
  \item unlimited growth policy: cache increases on every write
  \item \(n\) key limit policy: cache constrained to \(n\) keys
\end{itemize}

We trigger the policy switch at 100K keys to absorb the flash crowd at the
beginning of the run. Once triggered, keys are evicted to bring the size of the
cache down to the threshold and the least recently used keys are evicted.
In that bar chart, the cache sizes are along the \(x\) axis.

% results: same level of performance can be achieved 
The dynamic policies show better performance than the single \(n\) key
policies. The performance and memory utilization for a 100K key cache size is
the same as the 100K bar in the middle graphs but the rest reduce the size of
the keyspace after the read flash crowd. This reduces read/write traffic on the
persistent database and lowers the number of stalls.  We see the worst
performance when the engine switches to the 10 key limit policy, which achieves
94\% of the performance while only using 40KB of memory. 

% caveats: it is calculating 90% of the trajectory, memory value reported is final
\subsubsection*{Caveats}

The results in Figure~\ref{fig:methodology-tradeoff-dynamic} are slightly
deceiving for three reason: (1) segments take longer to generate later in the
run, (2) the memory footprint is the value at the end of 2.5 hours, and (3)
this policy only works well for the 2.5 hour run.  For (1), the curving down of
the simulation vs. wall-clock time is shown in
Figure~\ref{fig:methodology-trajectory}; as the nanoparticle grows it takes
longer to generate segments so by the time we reach 2 hours, over 90\% of the
trajectory is already generated.  For (2), the memory footprint is around 0.4GB
until we reach 100K key threshold. In
Figures~\ref{fig:methodology-tradeoff}
and~\ref{fig:methodology-tradeoff-dynamic} we plot the final value. For (3),
Figure~\ref{fig:methodology-trajectory} shows that the cache fills up with 100K
keys at time 7200 seconds and its size is reduced to the size listed in the
legend.  The curves stay close to ``Unlimited" for up to an hour after the
cache is reduced but eventually flatten out as the persistent database gets
overloaded. 10K and 100K follow the ``Unlimited" curve the longest and are
sufficient policies for the 2.5 hour runs but anything longer would need a
different dynamic load balancing policy.

\begin{figure}[t]
  \noindent\includegraphics[height=4.5cm,width=0.4\textwidth]{figures/methodology-trajectory.png}\\
  \caption{The rate that the trajectory is computed decays over time (which is
  expected) but this skews the performance improvements in
  Figure~\ref{fig:methodology-tradeoff-dynamic}. Our dynamic policy works for 2.5
  hour jobs but not for 4 hour jobs.  \label{fig:methodology-trajectory}}
\end{figure}

% but the result is still valid
Despite these caveats, the result is still valid: we found a dynamic load
balancing policy that absorbs the cost of a high read throughput on a small
keyspace and reduces the memory pressure for a 2.5 hour run. Our experiments
show the effectiveness of the load balancing policy engine we integrated into
ParSplice, not that we were able to identify the best policy for all system
setups ({\it i.e.} different ParSplice parameters, number of worker tasks, and
job lengths).  To solve that problem, we need a way to identify what thresholds
we should use for different job permutations.



\section{Using ML for the Keyspace}
\label{sec:ml-for-the-keyspace}

We proved in the previous section that an optimal policy, based on the read
burstiness at the beginning of the run,  exists for a 100K growth rate on 8
nodes, but we cannot re-do this analysis for every workload, system, and
parameter permutation.  Fortunately,
Section~\S\ref{sec:parsplice-keyspace-analysis} shows that the keyspace size
and activity is structured. Rather than finding policies by hand again for
every cluster size, growth rate, and temperature, in this section we use
machine learning to inform the Mantle policy engine.  Machine learning is good
at two things: handling large design spaces and matching patterns. Our keyspace
analysis in Section~\ref{sec:parsplice-keyspace-analysis} demonstrated a large
design space and Figure~\ref{fig:motivation-regimes} shows 4 workload regimes:
one plateau of redundant reads at the beginning, decreasing requests per
second, and then two plateaus of steady requests per second. We start with a
simple clustering algorithm to attack this multi-dimensional design space
problem and detect workload regimes.

% implementation: 
We feed the read request rate from the 100K and 1M runs in
Figure~\ref{fig:motivation-regimes} into the K-means clustering algorithm as
(timestamp, ops/second) tuples\footnote{Note that the magnitudes are different
because Figure~\ref{fig:motivation-regimes} was run with keyspace tracing on,
which reduces performance}. We weight the timestamp and ops/second equally and
set the number of clusters to be 4.  We chose this initial K  based on visual
inspection of Figure~\ref{fig:motivation-regimes} Knowing that the setup
parameters transform the request rates temporally or spatially, this same
initial K should work for all setups. Once the algorithm identifies the
workload regimes, we select the start of the third regime as the point to
switch to a fixed sized cache because the request rate has lowered to
sustainable levels for the persistent database.

We plot throughput over time in Figure~\ref{fig:futurework-regimes} and
color each point with its assigned group. The black stars are the centroids,
also known as the center of K-means groups.  We run the algorithm for a variety
of request rate traces but only show the setups from
Figure~\ref{fig:motivation-regimes}. We also annotate the graphs with the
suggested cache size, which is calculated by looking up the timestamp for the
third regime that corresponds to the keyspace size in our performance counters.

% results: 1. identifies 4 phases, 2. picks different timestamps for the third
% regime, suggests proper key values.
The algorithm properly identifies the 4 workload phases: the plateau of
redundant reads, the phase with a large decrease in request rate, and the two
plateaus of steady read requests. It also picks different timestamps for the
start of the third regime, which aligns with our keyspace analysis and our
assertion that the growth parameters affect how long it takes the workload to
reach a certain phase.  Finally, the algorithm picks reasonable values for the
key cache size. The 100K growth rate selects a 55K cache size, which is between
our benchmarked optimal threshold for the high watermark value chosen to
absorb the read burst (100K) and the lower cache size we limit the system to
after the initial burst (10K). This result both reaffirms the results from the
previous section and provides hope that we can avoid lengthy parameters sweeps
for ParSplice in the future.

\begin{figure}[t]
\noindent\includegraphics[width=0.4\textwidth]{figures/futurework-regimes.png}\\
  \caption{Learning the workload regimes with K-Means clustering helps pick
  keyspace size thresholds that can be fed into a dynamic load balancing policy
  engine, like Mantle. Specifying 4 clusters and selecting the third for
  informing the policy switch returns keyspace size thresholds similar to the
  values we found by hand in
  Section~\S\ref{sec:the-need-for-dynamic-load-balancing-policies}.
\label{fig:futurework-regimes}}
\end{figure}

\section{Related Work}

Key-value storage organizations for scientific applications is a field gaining
rapid interest. In particular, the analysis of the ParSplice keyspace and the
development of an appropriate scheme for load balancing is a direct response to
a case study for computation caching in scientific
applications~\cite{jenkins:ipdsw17-mochi}. In that work the authors motivated
the need for a flexible load balancing \emph{microservice} to efficiently scale
a memoization microservice. Our work is also heavily influenced by the
Malacology project~\cite{sevilla:eurosys17-malacology} which seeks to provide
fundamental services from within the storage system ({\it e.g.}, consensus) to
the application.

State-of-the-art distributed file systems partition write-heavy workloads and
replicate read-heavy workloads, similar to the approach we are advocating
here.  IndexFS~\cite{ren:sc2014-indexfs} partitions directories and clients
write to different partitions by grabbing leases and caching ancestor metadata
for path traversal. ShardFS takes the replication approach to the extreme by
copying all directory state to all nodes. The Ceph file system
(CephFS)~\cite{weil:sc2004-dyn-metadata, weil:osdi2006-ceph} employs both
techniques to a lesser extent; directories can be replicated or sharded but the
caching and replication policies are controlled with tunable parameters.  These
systems still need to be tuned by hand with {\it ad-hoc} policies designed for
specific applications.  Setting policies for migrations is arguably more
difficult than adding the migration mechanisms themselves.  For example,
IndexFS/CephFS use the GIGA+~\cite{patil:fast2011-giga} technique for
partitioning directories at a \emph{predefined} threshold. Mantle makes headway
in this space by providing a framework for exploring these policies, but does
not attempt anything more sophisticated (e.g., machine learning) to create
these policies. 

% ml and autotuning
Auto-tuning is a well-known technique used in
HPC~\cite{behzad:sc2013-autotuning, behzad:techreport2014-io-autotuning}, big
data systems systems~\cite{herodotou_starfish_2011}, and
databases~\cite{schnaitter_index_2009}.  Like our work, these systems focus on
the physical design of the storage ({\it e.g.} cache size) but since we focused
on a relatively small set of parameters (cache size, migration thresholds), we
did not need anything as sophisticated as the genetic algorithm used
in~\cite{behzad:sc2013-autotuning}.  We cannot drop these techniques into
ParSplice because the magnitude and speed of the workload hotspots/flash crowds
makes existing approaches less applicable. 

Our plan is to use MDHIM~\cite{greenberg:hotstorage2015-mdhim} as our back-end
key-value store because it was designed for HPC and has the proper mechanisms
for migration already implemented.  

\section{Conclusion}

Load balancing is a well-known technique for alleviating load and improving
performance, yet finding the best policies for applications is still a
hands-on, tedious process. If a load balancing module were to be accepted into
a suite of services like the Mochi project, it must transcend domains and be
useful to a wide-range of applications. In this paper, we present the framework
for such a load balancing service which (1) helps administrators explore the
techniques for informing load balancing, (2) supports dynamic policies for
quickly changing workloads and higher level intelligence.  We demonstrate (1)
with a keyspace analysis for ParSplice and use our findings to design a load
balancing policy that improves resource utilization.  To drive the policy
engine designed in (2), we show that a single policy is inadequate and that
Mantle is flexible enough to explore policies generated with machine learning.
It is our hope that the introduction of the Mantle approach encourages the use
of machine learning and auto-tuning for policy design in future storage
systems.


\bibliographystyle{ACM-Reference-Format}
\bibliography{references} 

\end{document}
