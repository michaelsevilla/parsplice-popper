\documentclass[sigconf]{acmart}
\usepackage{arydshln}

\usepackage{booktabs} % For formal tables


% Copyright
\setcopyright{none}
\settopmatter{printacmref=false}
%\setcopyright{acmcopyright}
%\setcopyright{acmlicensed}
%%\setcopyright{rightsretained}
%\setcopyright{usgov}
%\setcopyright{usgovmixed}
%\setcopyright{cagov}
%\setcopyright{cagovmixed}
\setlength{\belowcaptionskip}{-17pt}

%% DOI
%\acmDOI{10.475/123_4}
%
%% ISBN
%\acmISBN{123-4567-24-567/08/06}
%
%%:Conference
\acmConference[]{}{}{}
%\copyrightyear{2016}
%
%\acmArticle{4}
%\acmPrice{15.00}

% These commands are optional
%\acmBooktitle{Transactions of the ACM Woodstock conference}
%\editor{Jennifer B. Sartor}
%\editor{Theo D'Hondt}
%\editor{Wolfgang De Meuter}


\begin{document}
\title{ParSplice Keyspace Locality}
%\titlenote{Produces the permission block, and
%  copyright information}
%\subtitle{Extended Abstract}
%\subtitlenote{The full version of the author's guide is available as
%  \texttt{acmart.pdf} document}
%`
%`\author{Michael A. Sevilla, Carlos Maltzahn}
%`\affiliation{\institution{University of California, Santa Cruz}}
%`\email{{msevilla, carlosm}@soe.ucsc.edu}
%`
%`\makeatother
%`\author{Bradley W. Settlemyer, Danny Perez, David Rich, Galen M. Shipman (LANL)}
%`\email{{bws, danny_perez, dor, gshipman}@lanl.gov}
%`
%`\begin{abstract}
%`
%`Our analysis of the key-value activity generated by the ParSplice molecular
%`dynamics simulation demonstrates the need for a distributed, load balancing
%`key-value store.  We observe clear access regimes and hot spots that offer
%`significant opportunity for optimization. We leverage the Mantle load balancing
%`framework, which was originally designed for distributed file systems, to
%`dynamically switch policies and present a two policy scheme that achieves 96\%
%`efficiency while using only 7.6\% of the memory resources required by the base
%`case. Finally, we demonstrate how a machine learning clustering technique is an
%`effective method for detecting access patterns within the keyspace over time.  
%`
%`\end{abstract}

%
% The code below should be generated by the tool at
% http://dl.acm.org/ccs.cfm
% Please copy and paste the code instead of the example below. 
%
%\begin{CCSXML}
%<ccs2012>
% <concept>
%  <concept_id>10010520.10010553.10010562</concept_id>
%  <concept_desc>Computer systems organization~Embedded systems</concept_desc>
%  <concept_significance>500</concept_significance>
% </concept>
% <concept>
%  <concept_id>10010520.10010575.10010755</concept_id>
%  <concept_desc>Computer systems organization~Redundancy</concept_desc>
%  <concept_significance>300</concept_significance>
% </concept>
% <concept>
%  <concept_id>10010520.10010553.10010554</concept_id>
%  <concept_desc>Computer systems organization~Robotics</concept_desc>
%  <concept_significance>100</concept_significance>
% </concept>
% <concept>
%  <concept_id>10003033.10003083.10003095</concept_id>
%  <concept_desc>Networks~Network reliability</concept_desc>
%  <concept_significance>100</concept_significance>
% </concept>
%</ccs2012>  
%\end{CCSXML}

%\ccsdesc[500]{Computer systems organization~Embedded systems}
%\ccsdesc[300]{Computer systems organization~Redundancy}
%\ccsdesc{Computer systems organization~Robotics}
%\ccsdesc[100]{Networks~Network reliability}
%
%
%\keywords{ACM proceedings, \LaTeX, text tagging}


\maketitle

\section{Outline}


% https://en.wikipedia.org/wiki/Scientific_method

\subsection{Formulation of a Question: Do HPC kv applications need dynamic load
balancing policies or is there a one-size-fits-all policy?}

\subsection{Hypothesis: The Mantle approach is an effective mechanism for
switching load balancing policies in HPC key-value applications}

\begin{itemize}
  \item Mantle is a load balancing approach and API.
  \begin{itemize}
    \item quantifies effect of load balancing
    \item formalized effective FS balancers
    \item debugging tool 
  \end{itemize}
  \item HXHIM has migration mechanisms for load balancing
  \begin{itemize}
    \item bulk operations (\texttt{put/get()})
    \item key partitioners
    \item secondary indices
  \end{itemize}
\end{itemize}

\subsection{Prediction}

In order from most likely to least likely:

\begin{enumerate}

  \item HPC key-value store workloads are structured (because they are mostly
  workflows and simulations) that their job phases can be learned and exploited
  using dynamic load balancing policies.

  \item HPC key-value store workloads are so structured that one
  policy-fits-all

  \item HPC key-value store workloads are not structured enough to be learned

  \item HPC key-value store workload hotspots/flash crowds are too fast to be
  exploited

\end{enumerate}

\subsection{Testing: Combine Mantle and HXHIM to explore dynamic load balancing
policies for ParSplice, an HPC application with distinct workload phases}

\subsubsection{ParSplice is structured}
Figure~\ref{fig:scale-length}.
\begin{figure*}[tbh]
  \noindent\includegraphics[width=1\textwidth]{figures/scale-length.png}\\
  \caption{We can predict how fast the keyspace grows and which parts of the
  namespace are popular.\label{fig:scale-length}}
\end{figure*}


\subsubsection{ParSplice behavior can change}

Figure~\ref{fig:scale-delay}.
\begin{figure}[tbh]
  \noindent\includegraphics[width=0.5\textwidth]{figures/scale-delay.png}\\
  \caption{With a simple parameter tweak, the workload regimes changes
  (timescale) so one size does not fit all.  \label{fig:scale-delay}}
\end{figure}



\subsection{Analsyis}

\pagebreak


\bibliographystyle{ACM-Reference-Format}
\bibliography{references} 

\end{document}
